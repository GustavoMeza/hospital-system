\hypertarget{documentation}{%
\section{Documentation}\label{documentation}}

\hypertarget{database-schema}{%
\subsection{Database schema}\label{database-schema}}

\textbf{batches}

\begin{itemize}
\item
  Drugs are acquired in batches
\item
  Drugs in the same batch share the same expiration date.
\end{itemize}

\textbf{drugs}

\begin{itemize}
\item
  Different types of drugs the hospital has.
\item
  Different presentations are considered different drugs.
\end{itemize}

\textbf{fill\_batches}

\begin{itemize}
\tightlist
\item
  Outputs caused by a prescription fill
\end{itemize}

\textbf{inputs}

\begin{itemize}
\item
  Drug acquisition
\item
  Can either be by purchase, transfer or other.
\item
  In one input we can obtain multiple batches.
\end{itemize}

\textbf{licenses}

\begin{itemize}
\item
  Different licenses that Doctors have attained through their career
\item
  Can either be professional licenses or specialties.
\end{itemize}

\textbf{outputs}

\begin{itemize}
\item
  Drugs leaving the hospital
\item
  Can either be because they expired, we filled a prescription or other.
\item
  In one output we group drugs of the same batch.
\item
  Needs to be signed by a manager.
\end{itemize}

\textbf{patients}

\begin{itemize}
\tightlist
\item
  The patients at the hospital.
\end{itemize}

\textbf{permissions}

\begin{itemize}
\tightlist
\item
  Ability to perform an action on the API
\end{itemize}

\textbf{prescription\_drugs}

\begin{itemize}
\item
  Drugs required in a prescription
\item
  Intake indications.
\end{itemize}

\textbf{prescription\_fills}

\begin{itemize}
\tightlist
\item
  Information when a prescription is filled.
\end{itemize}

\textbf{prescription\_returns}

\begin{itemize}
\tightlist
\item
  Information when some prescribed drugs are returned
\end{itemize}

\textbf{prescriptions}

\begin{itemize}
\tightlist
\item
  Prescription issued by a doctor to a patient.
\end{itemize}

\textbf{purchases}

\begin{itemize}
\tightlist
\item
  Additional information required if the input is a purchase.
\end{itemize}

\textbf{role\_permissions}

\begin{itemize}
\tightlist
\item
  Permissions that a certain role has.
\end{itemize}

\textbf{roles}

\begin{itemize}
\tightlist
\item
  Different roles inside the system.
\end{itemize}

\textbf{specialties}

\begin{itemize}
\tightlist
\item
  Additional information required if the license is a specialty
\end{itemize}

\textbf{transfers}

\begin{itemize}
\tightlist
\item
  Additional information required if the input is a transfer
\end{itemize}

\textbf{user\_roles}

\begin{itemize}
\tightlist
\item
  Assignation of users and their roles.
\end{itemize}

\textbf{users}

\begin{itemize}
\tightlist
\item
  Users of the system.
\end{itemize}

You can see in more detail the relationships \href{'./schema.pdf'}{here}

\hypertarget{database-audit}{%
\subsection{Database audit}\label{database-audit}}

We want to be able to audit changes in the database to track unwanted
behaviours To be able to make this audits, we keep track of all the
changes in an entity. Our solution has two parts:

\begin{itemize}
\item
  Persistent operations, this means to save the current state of an
  entity before changing it. This operations are handled by the API.
\item
  Metadata of the changes made:
\item
  \textbf{created\_at} The moment of the last change to this record.
\item
  \textbf{created\_by} The user responsible of the last change to this
  record.
\item
  \textbf{original\_id} Since we are saving snapshots of the entities,
  these are going to have new ids. This field stores the id of the
  original object.
\item
  \textbf{status}: Either \texttt{active} if the records belongs to the
  current state of the database or \texttt{history} if the record
  belongs to an older state of the database.
\end{itemize}

\href{https://www.codeproject.com/Articles/105768/Audit*Trail\%20-Tracing-Data-Changes-in-Database}{This
article} explains our approach in more detail.
